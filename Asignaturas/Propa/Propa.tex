\documentclass[10pt]{book}



\usepackage{amsmath}
\usepackage{amssymb}
\usepackage{amsthm}
\usepackage{array}
\usepackage{babelbib}
\usepackage{braket}
\usepackage{caption}
\usepackage{colortbl}
\usepackage{rotating}
\usepackage[table]{xcolor}
\usepackage{color}
\usepackage{enumerate}
\usepackage{esint}
\usepackage{eso-pic}
\usepackage{listings}
\usepackage{lscape}
\usepackage{mathtools}
\usepackage{multicol}
\usepackage{multirow}
\usepackage{siunitx}
\usepackage{subcaption}
\usepackage{subdepth}
\usepackage{tcolorbox}
\usepackage{tikz}
\usepackage{titlesec}
\usepackage{titling}
\usepackage{upgreek}
\usepackage{url}
\usepackage{verbatim}
\usepackage{vwcol}
\usepackage{wallpaper}
\usepackage{xfrac}
\usepackage{physics}
\usepackage[c]{esvect}
\usepackage[utf8]{inputenc}
\usepackage[fleqn]{nccmath}
\usepackage[thicklines]{cancel}
\usepackage[margin=2cm]{geometry}
\usepackage[colorlinks=true,spanish]{hyperref}
\usepackage[oldvoltagedirection]{circuitikz}
\usepackage[greek,spanish,es-tabla,es-nodecimaldot,es-noindentfirst]{babel}
\usepackage[symbol]{footmisc}
\renewcommand{\thefootnote}{\fnsymbol{footnote}}

\sisetup{
  per-mode = fraction,
  detect-all,
  exponent-product = \cdot
}
\hypersetup{
  citecolor = blue,
  linkcolor = blue,
  urlcolor = blue,
  pdfauthor = {Javier Rodrigo López}
}
\captionsetup[figure]{labelfont={bf},name={Figura},labelsep=period}
\captionsetup[table]{labelfont={bf},name={Tabla},labelsep=period}
\titleformat{\section}{\normalfont\Large\bfseries}{\thesection}{1em}{}[{\titlerule[0.8pt]}]
\titleformat{\subsubsection}{\normalfont\normalsize\bfseries}{\thesubsubsection}{1em}{}[{\titlerule[0.05pt]}]
\titlespacing{\section}{0pt}{2\parskip}{\parskip}
\titlespacing{\subsection}{0pt}{\parskip}{0pt}
\titlespacing{\subsubsection}{0pt}{\parskip}{0pt}
\usepackage{enumitem}
\setlist{before={\parskip=3pt}, after=\vspace{\baselineskip}}
\setlength{\parindent}{0pt}
\setlength{\parskip}{0.5em}

\usepackage{booktabs}
\usepackage{bigstrut}

\renewcommand{\vec}{\vv}

% Tipografía
\renewcommand{\familydefault}{\sfdefault}
\renewcommand{\rmdefault}{\sfdefault}

% Para escribir decibelios SPL
\DeclareSIUnit\dbspl{dB\ensuremath{_{\textnormal{SPL}}}}
\DeclareSIUnit\dBlin{dB\ensuremath{_{\textnormal{Lin}}}}
\DeclareSIUnit\dBA{dB\ensuremath{_{\textnormal{A}}}}
 % Se incluye el preámbulo

\title{\Huge Propagación de ondas\\\huge Apuntes de clase}
\author{Javier Rodrigo López}
\date{\today}

\begin{document}

% Este comando crea el título
\maketitle

% Este comando crea el índice
\tableofcontents


\newpage

\chapter*{Presentación}

Pablo Merodio (grupo de tarde) \href{mailto:pablo.merodio@upm.es}{pablo.merodio@upm.es} a Pilar, explicando exhaustivamente el motivo. Tendrán una duración máxima de 20 minutos. El horario de las tutorías se puede encontrar en Moodle o en la Intranet. Hay que indicar 3 opciones para las tutorías.

Hay que animarse a contestar las dudas de foro, aunque después de un tiempo contestarán los profesores.

Primer parcial (7 de noviembre), tiene nota mínima de 3/10. Según la nota que se obtenga, se puede decidir presentarse al examen global en lugar del segundo parcial.

El examen extraordinario no ``guarda'' parciales. Es decir, cae el temario completo y nota mínima de 5/10.

\newpage

\chapter{Operadores vectoriales}
\section{Gradiente de un campo escalar}

\section{Sistemas de coordenadas ortogonales}

Los sistemas de coordenadas sirven para caracterizar unívocamente cualquier punto del espacio mediante una terna de números.

Los \textbf{sistemas ortogonales} son los sistemas de vectores unitarios (de módulo 1) que son perpendiculares entre sí.

Las coordenadas cartesianas son representadas por $\left( x,y,z \right)$.
\[\left\lbrace \begin{matrix}
    -\infty < x < \infty \\
    -\infty < y < \infty \\
    -\infty < z < \infty
  \end{matrix} \right.\]

\[ \vec{r} = x \vv*{u}{x} + y \vv*{u}{y} + \vv*{u}{z} \]

Donde $\vv*{u}{x}$ es un vector unitario en la dirección del eje $x$, y viceversa.

\begin{align*}
                        & \text{Diferencial de longitud}   & \vec{\dd l}  & = \dd x \vv*{u}{x} + \dd y \vv*{u}{y} + \dd z \vv*{u}{z} \\
                        & \text{Diferencial de volumen}    & \vec{\dd v}  & = \dd x  \dd y  \dd z                                    \\[5 pt]
                        & \text{Diferencial de superficie} & \left\lbrace
  \begin{matrix*}[l]
    x = \text{cte.} \\
    y = \text{cte.} \\
    z = \text{cte.}
  \end{matrix*} \right. &
  \begin{matrix*}[l]
    \quad \rightarrow \vec{\dd S} = \pm \dd y \dd z \vv*{u}{x}\\
    \quad \rightarrow \vec{\dd S} = \pm \dd x \dd z \vv*{u}{y}\\
    \quad \rightarrow \vec{\dd S} = \pm \dd x \dd y \vv*{u}{z}
  \end{matrix*}                                                                          \\
\end{align*}

El sistema de \textbf{coordenadas cilíndricas}. $\rho$ es la distancia al eje $z$. $\varphi$ es el ángulo con respecto al semiplano $xz$ con $x$ positivo.

\[\left\lbrace \begin{matrix}
    0 < \rho < \infty \\
    0 \leq y < 2\pi   \\
    -\infty < z < \infty
  \end{matrix} \right. \]

El vector unitario $\vv*{u}{z}$ es igual que en las coordenadas cartesianas. Sin embargo, debemos definir los otros dos. El vector $\vv*{u}{\rho}$

\[ \left.
  \begin{matrix}
    -\infty & < x & < \infty \\
    -\infty & < y & < \infty \\
    -\infty & < z & < \infty
  \end{matrix} \right\rbrace \]

\begin{align*}
                           & \text{Diferencial de longitud}   & \vec{\dd l}  & = \dd \rho \vv*{u}{\rho} + \rho\dd \varphi \vv*{u}{\varphi} + \dd z \vv*{u}{z} \\
                           & \text{Diferencial de volumen}    & \vec{\dd v}  & = \rho \dd \rho \dd\varphi \dd z                                               \\[5 pt]
                           & \text{Diferencial de superficie} & \left\lbrace
  \begin{matrix*}[l]
    \rho = \text{cte.} \\
    \varphi = \text{cte.} \\
    z = \text{cte.}
  \end{matrix*} \right. &
  \begin{matrix*}[l]
    \quad \rightarrow \vec{\dd S} = \pm \rho \dd \varphi \dd z \vv*{u}{}\\
    \quad \rightarrow \vec{\dd S} = \pm \dd x \dd z \vv*{u}{y}\\
    \quad \rightarrow \vec{\dd S} = \pm \dd x \dd y \vv*{u}{z}
  \end{matrix*}                                                                                         \\
\end{align*}

El sistema de \textbf{coordenadas esféricas}. $r$ es la distancia al origen de coordenadas. $\rho$ es el ángulo con respecto al eje $xz$ con $x$ positivo.

\[\left\lbrace \begin{matrix}
    0 < r < \infty      \\
    0 \leq \theta < \pi \\
    0 \leq \varphi < 2\pi
  \end{matrix} \right. \]

El vector unitario $\vv*{u}{z}$ es igual que en las coordenadas cartesianas. Sin embargo, debemos definir los otros dos. El vector $\vv*{u}{\rho}$

\[ \left.
  \begin{matrix}
    -\infty & < x & < \infty \\
    -\infty & < y & < \infty \\
    -\infty & < z & < \infty
  \end{matrix} \right\rbrace \]

El vector $\vv*{u}{\theta}$ se dibuja como perpendicular a $\vv*{u}{r}$, con dirección al eje $z$ y en sentido de crecimiento positivo.

\begin{align*}
                           & \text{Diferencial de longitud}   & \vec{\dd l}  & = \dd r \vv*{u}{r} + r\dd \theta \vv*{u}{\theta} +r \sen \left( \theta \right) \dd \theta \vv*{u}{\theta} \\
                           & \text{Diferencial de volumen}    & \vec{\dd v}  & = r^2 \sen \left( \theta \right) \dd                                                                      \\[5 pt]
                           & \text{Diferencial de superficie} & \left\lbrace
  \begin{matrix*}[l]
    \rho = \text{cte.} \\
    \varphi = \text{cte.} \\
    z = \text{cte.}
  \end{matrix*} \right. &
  \begin{matrix*}[l]
    \quad \rightarrow \vec{\dd S} = \pm \rho \dd \varphi \dd z \vv*{u}{}\\
    \quad \rightarrow \vec{\dd S} = \pm \dd x \dd z \vv*{u}{y}\\
    \quad \rightarrow \vec{\dd S} = \pm \dd x \dd y \vv*{u}{z}
  \end{matrix*}                                                                                                                    \\
\end{align*}

Para pasar de coordenadas cartesianas a cilíndricas:
\begin{enumerate}
  \item $z = z$
  \item $ 3$
\end{enumerate}

Algo es \textbf{constante} si no varía en el tiempo. Algo es \textbf{uniforme} si no varía en el espacio.

\section{Operador Nabla}
\section{Gradiente de un campo escalar}
\section{Algo más...}
\section{Divergencia y rotacional de un campo vectorial}
\section{Teorema de Helmholzt}
\chapter{Ondas acústicas planas}
\chapter{Ondas acústicas esféricas}
\chapter{Reflexión y refracción de ondas acústicas}



\end{document}
