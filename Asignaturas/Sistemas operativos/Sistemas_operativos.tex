\documentclass[10pt]{book}



\usepackage{amsmath}
\usepackage{amssymb}
\usepackage{amsthm}
\usepackage{array}
\usepackage{babelbib}
\usepackage{braket}
\usepackage{caption}
\usepackage{colortbl}
\usepackage{rotating}
\usepackage[table]{xcolor}
\usepackage{color}
\usepackage{enumerate}
\usepackage{esint}
\usepackage{eso-pic}
\usepackage{listings}
\usepackage{lscape}
\usepackage{mathtools}
\usepackage{multicol}
\usepackage{multirow}
\usepackage{siunitx}
\usepackage{subcaption}
\usepackage{subdepth}
\usepackage{tcolorbox}
\usepackage{tikz}
\usepackage{titlesec}
\usepackage{titling}
\usepackage{upgreek}
\usepackage{url}
\usepackage{verbatim}
\usepackage{vwcol}
\usepackage{wallpaper}
\usepackage{xfrac}
\usepackage{physics}
\usepackage[c]{esvect}
\usepackage[utf8]{inputenc}
\usepackage[fleqn]{nccmath}
\usepackage[thicklines]{cancel}
\usepackage[margin=2cm]{geometry}
\usepackage[colorlinks=true,spanish]{hyperref}
\usepackage[oldvoltagedirection]{circuitikz}
\usepackage[greek,spanish,es-tabla,es-nodecimaldot,es-noindentfirst]{babel}
\usepackage[symbol]{footmisc}
\renewcommand{\thefootnote}{\fnsymbol{footnote}}

\sisetup{
  per-mode = fraction,
  detect-all,
  exponent-product = \cdot
}
\hypersetup{
  citecolor = blue,
  linkcolor = blue,
  urlcolor = blue,
  pdfauthor = {Javier Rodrigo López}
}
\captionsetup[figure]{labelfont={bf},name={Figura},labelsep=period}
\captionsetup[table]{labelfont={bf},name={Tabla},labelsep=period}
\titleformat{\section}{\normalfont\Large\bfseries}{\thesection}{1em}{}[{\titlerule[0.8pt]}]
\titleformat{\subsubsection}{\normalfont\normalsize\bfseries}{\thesubsubsection}{1em}{}[{\titlerule[0.05pt]}]
\titlespacing{\section}{0pt}{2\parskip}{\parskip}
\titlespacing{\subsection}{0pt}{\parskip}{0pt}
\titlespacing{\subsubsection}{0pt}{\parskip}{0pt}
\usepackage{enumitem}
\setlist{before={\parskip=3pt}, after=\vspace{\baselineskip}}
\setlength{\parindent}{0pt}
\setlength{\parskip}{0.5em}

\usepackage{booktabs}
\usepackage{bigstrut}

\renewcommand{\vec}{\vv}

% Tipografía
\renewcommand{\familydefault}{\sfdefault}
\renewcommand{\rmdefault}{\sfdefault}

% Para escribir decibelios SPL
\DeclareSIUnit\dbspl{dB\ensuremath{_{\textnormal{SPL}}}}
\DeclareSIUnit\dBlin{dB\ensuremath{_{\textnormal{Lin}}}}
\DeclareSIUnit\dBA{dB\ensuremath{_{\textnormal{A}}}}
 % Se incluye el preámbulo

\title{\Huge Sistemas operativos\\\huge Apuntes de clase}
\author{Javier Rodrigo López}
\date{\today}

\begin{document}

% Este comando crea el título
\maketitle

% Este comando crea el índice
\tableofcontents


\newpage

\newpage

\chapter{Introducción a los sistemas operativos}

\section{Visión completa de un ordenador}

El sistema operativo es el único software que puede interactuar con el hardware. El usuario interactúa con las aplicaciones, las cuales utilizan al sistema operativo como intermediario entre ellas y el hardware.

Entre las utilidades que tiene el sistema operativo, encontramos:
\begin{itemize}
  \item Ocultar el hardware y, por tanto, la complejidad que tiene su uso.
  \item Operar como gestor de recursos. Con recursos, nos referimos a la memoria, el procesador, dispositivos de entrada/salida...
  \item Permite la multiplexación en tiempo y espacio.
  \item Gestión eficiente de los conflictos entre procesos.
  \item Control del uso indebido del hardware.
\end{itemize}

A pesar de esta larga enumeración, el objetivo principal de un sistema operativo es \textbf{gestionar los recursos del sistema informático}. Como objetivos secundarios, que tenga una interfaz de uso amigable, que sea eficiente, fiable...

El núcleo del sistema operativo, también llamado \textbf{kernel}, es el que gestiona todos los recursos. Los \textbf{programas del sistema} son los programas que incluye el fabraciante del sistema operativo para facilitar el uso a los usuarios. El \textbf{shell}, es la capa de aplicaciones con las que interactúa el usuario. Pueden hacer uso de una CLI (\textit{Command Line Interface}), una GUI (\textit{Graphic User Interface}) o una interfaz por lotes \textit{batch}.

\section{Procesos}

\subsection{Concepto de proceso}

Un proceso es un programa en ejecución con un contexto que gestiona el núcleo. El contexto hace referencia a la información que necesita saber el sistema operativo para trabajar con él, como los registros que se están usando, la memoria donde está almacenado el código del proceso, la comunicación con otros procesos, un número de identificación... El contexto se guarda en lo que se denomina PCB. Hay un PCB por cada proceso.

\chapter{Gestión del procesador}


\chapter{Gestión de la memoria}


\chapter{Concurrencia}


\chapter{Gestión de la entrada/salida}


\chapter{Sistemas de ficheros}

\appendix


\chapter{Instalación de FreeBSD sobre VMware}

\chapter{Entorno de trabajo: mandatos comunes de Unix + programación}

\chapter{Manejo de procesos (en C)}

\chapter{Hilos y concurrencia (en Java)}

\end{document}
