\documentclass[10pt]{book}



\usepackage{amsmath}
\usepackage{amssymb}
\usepackage{amsthm}
\usepackage{array}
\usepackage{babelbib}
\usepackage{braket}
\usepackage{caption}
\usepackage{colortbl}
\usepackage{rotating}
\usepackage[table]{xcolor}
\usepackage{color}
\usepackage{enumerate}
\usepackage{esint}
\usepackage{eso-pic}
\usepackage{listings}
\usepackage{lscape}
\usepackage{mathtools}
\usepackage{multicol}
\usepackage{multirow}
\usepackage{siunitx}
\usepackage{subcaption}
\usepackage{subdepth}
\usepackage{tcolorbox}
\usepackage{tikz}
\usepackage{titlesec}
\usepackage{titling}
\usepackage{upgreek}
\usepackage{url}
\usepackage{verbatim}
\usepackage{vwcol}
\usepackage{wallpaper}
\usepackage{xfrac}
\usepackage{physics}
\usepackage[c]{esvect}
\usepackage[utf8]{inputenc}
\usepackage[fleqn]{nccmath}
\usepackage[thicklines]{cancel}
\usepackage[margin=2cm]{geometry}
\usepackage[colorlinks=true,spanish]{hyperref}
\usepackage[oldvoltagedirection]{circuitikz}
\usepackage[greek,spanish,es-tabla,es-nodecimaldot,es-noindentfirst]{babel}
\usepackage[symbol]{footmisc}
\renewcommand{\thefootnote}{\fnsymbol{footnote}}

\sisetup{
  per-mode = fraction,
  detect-all,
  exponent-product = \cdot
}
\hypersetup{
  citecolor = blue,
  linkcolor = blue,
  urlcolor = blue,
  pdfauthor = {Javier Rodrigo López}
}
\captionsetup[figure]{labelfont={bf},name={Figura},labelsep=period}
\captionsetup[table]{labelfont={bf},name={Tabla},labelsep=period}
\titleformat{\section}{\normalfont\Large\bfseries}{\thesection}{1em}{}[{\titlerule[0.8pt]}]
\titleformat{\subsubsection}{\normalfont\normalsize\bfseries}{\thesubsubsection}{1em}{}[{\titlerule[0.05pt]}]
\titlespacing{\section}{0pt}{2\parskip}{\parskip}
\titlespacing{\subsection}{0pt}{\parskip}{0pt}
\titlespacing{\subsubsection}{0pt}{\parskip}{0pt}
\usepackage{enumitem}
\setlist{before={\parskip=3pt}, after=\vspace{\baselineskip}}
\setlength{\parindent}{0pt}
\setlength{\parskip}{0.5em}

\usepackage{booktabs}
\usepackage{bigstrut}

% Tipografía
\renewcommand{\familydefault}{\sfdefault}
\renewcommand{\rmdefault}{\sfdefault}

% Para escribir decibelios SPL
\DeclareSIUnit\dbspl{dB\ensuremath{_{\textnormal{SPL}}}}
\DeclareSIUnit\dBlin{dB\ensuremath{_{\textnormal{Lin}}}}
\DeclareSIUnit\dBA{dB\ensuremath{_{\textnormal{A}}}}
 % Se incluye el preámbulo

\title{\Huge Sistemas audiovisuales\\\huge Apuntes de clase}
\author{Javier Rodrigo López}
\date{\today}

\begin{document}

% Este comando crea el título
\maketitle

% Este comando crea el índice
\tableofcontents


\newpage

\section*{Presentación}

Elena Blanco, José Luis Rodríguez Vázquez.

Hacer uso de las tutorías.

\newpage



\chapter{Dispositivos de captación y reproducción de sonido e imagen}

\section{Micrófonos}

\section{Altavoces}

\section{Cámaras y sensores}

Los sistemas de vídeo van a convertir un escenario tridimensional con colores complejos en una señal unidimensional lo suficientemente parecida a la escena real como para ser entendida por el sistema visual humano.

Las frecuencias electromagnéticas que comprenden el espectro visible humano van de los \SI{400}{\nano\metre} a los \SI{750}{\nano\metre}. Por debajo se encuentra la radiación ultravioleta y por encima los infrarrojos.

El ojo tiene células sensibles a la intensidad luminosa (todos los colores) llamadas \textbf{bastones}, y otras células sensibles a los colores llamadas \textbf{conos}.

La \textbf{agudeza visual} es la capacidad que tiene una persona de observar dos objetos muy cercanos entre sí diferenciándolos como objetos independientes. Si adecuamos los píxeles de una pantalla, es posible dar una sensación de continuidad mediante el acercamiento de los píxeles entre sí, teniendo en cuenta la distancia a la que se encuentra el espectador.

Otra característica relacionada con la continuidad es la \textbf{memoria retentiva}. A partir de 12 o 14 imágenes por segundo, empezamos a percibir esa serie de imágenes como un vídeo contínuo.

El proceso de escaneo del sensor plano (bidimensional) es lo que nos permite transformar la información en una señal unidimensional.

\[ d = 42.5 \cdot \frac{D \text{ (pulgadas)}}{\text{nº de líneas}} \ \SI{}{\metre} \]

Aunque estuviera desarrollado el sistema de televisión en alta definición


\chapter{Señales y formatos de audio y vídeo}

\section{Digitalización de las señales de audio y vídeo}

\section{Codificación de la señal de audio}

\section{Codificación de la señal de vídeo}

\section{Soportes y formatos de almacenamiento de audio y vídeo}

\section{Transmisión de señales de audio y vídeo}



\chapter{Introducción a los sistemas de transmisión de vídeo y audio}

\section{Parámetros generales de un sistema de transmisión}

\section{Sistemas de transmisión por cable}

\section{Sistemas de transmisión por fibra óptica}

\section{Sistemas de transmisión y difusión terrestre}

\section{Sistemas de transmisión por satélite}


\end{document}
