\documentclass[10pt]{article}



\usepackage{amsmath}
\usepackage{amssymb}
\usepackage{amsthm}
\usepackage{array}
\usepackage{babelbib}
\usepackage{braket}
\usepackage{caption}
\usepackage{colortbl}
\usepackage{rotating}
\usepackage[table]{xcolor}
\usepackage{color}
\usepackage{enumerate}
\usepackage{esint}
\usepackage{eso-pic}
\usepackage{listings}
\usepackage{lscape}
\usepackage{mathtools}
\usepackage{multicol}
\usepackage{multirow}
\usepackage{siunitx}
\usepackage{subcaption}
\usepackage{subdepth}
\usepackage{tcolorbox}
\usepackage{tikz}
\usepackage{titlesec}
\usepackage{titling}
\usepackage{upgreek}
\usepackage{url}
\usepackage{verbatim}
\usepackage{vwcol}
\usepackage{wallpaper}
\usepackage{xfrac}
\usepackage{physics}
\AtBeginDocument{\RenewCommandCopy\qty\SI}
\usepackage[c]{esvect}
\usepackage[utf8]{inputenc}
\usepackage[fleqn]{nccmath}
\usepackage[thicklines]{cancel}
\usepackage[margin=2cm]{geometry}
\usepackage[colorlinks=true,spanish]{hyperref}
\usepackage[oldvoltagedirection]{circuitikz}
\usepackage[greek,spanish,es-tabla,es-nodecimaldot,es-noindentfirst]{babel}
\usepackage[symbol]{footmisc}
\renewcommand{\thefootnote}{\fnsymbol{footnote}}

\sisetup{
  per-mode = fraction,
  inline-per-mode = power,
  detect-all,
  exponent-product = \cdot
}
\hypersetup{
  citecolor = blue,
  linkcolor = blue,
  urlcolor = blue,
  pdfauthor = {Javier Rodrigo López}
}
\captionsetup[figure]{labelfont={bf},name={Figura},labelsep=period}
\captionsetup[table]{labelfont={bf},name={Tabla},labelsep=period}
\titleformat{\section}{\normalfont\Large\bfseries}{\thesection}{1em}{}[{\titlerule[0.8pt]}]
\titleformat{\subsection}{\normalfont\normalsize\bfseries}{\thesubsection}{1em}{}[{}]
\titlespacing{\section}{0pt}{2\parskip}{\parskip}
\titlespacing{\subsection}{0pt}{\parskip}{0pt}
\titlespacing{\subsubsection}{0pt}{\parskip}{0pt}
\usepackage{enumitem}
\setlist{before={\parskip=3pt}, after=\vspace{\baselineskip}}
\setlength{\parindent}{0pt}
\setlength{\parskip}{0.5em}
\renewcommand\thesubsection{\arabic{subsection}}
\renewcommand\thesubsubsection{\arabic{subsection}.\alph{subsubsection}}

\usepackage{booktabs}
\usepackage{bigstrut}

\renewcommand{\vec}{\vv}

% Tipografía
% \renewcommand{\familydefault}{\sfdefault}
% \renewcommand{\rmdefault}{\sfdefault}

% Para escribir decibelios SPL
\DeclareSIUnit\dbspl{dB\ensuremath{_{\textnormal{SPL}}}}
\DeclareSIUnit\dBlin{dB\ensuremath{_{\textnormal{Lin}}}}
\DeclareSIUnit\dBA{dB\ensuremath{_{\textnormal{A}}}}


\title{\Huge Práctica 2.2. Micrófonos \\\huge Laboratorio de Sistemas Electroacústicos}
\author{Javier Rodrigo López}
\date{\today}

\begin{document}
\maketitle
% \tableofcontents
\setcounter{subsection}{3}

\subsection{Adjuntar las gráficas de respuesta en frecuencia para el efecto proximidad y filtro eléctrico.}
\subsubsection{Representar vía Excel la gráfica de ganancia del efecto proximidad: $G _{\textnormal{prox}} [\unit{\dB} ] = S(\text{cerca}) [\unit{\dB} ]  – S(\text{lejos}) [\unit{\dB} ] $ (hay que compensar la diferencia de ganancia del preamplificador, si existió). La escala de frecuencias en \unit{\dB} desde \qty{20}{\hertz }.}
\subsubsection{Representar en Excel la respuesta del filtro paso alto que incorpora el micrófono de laboratorio, $H [\unit{\dB} ] $. La escala de frecuencias en \unit{\dB} desde \qty{20}{\hertz }. Determinar la frecuencia de corte a \qty{-3}{\dB} y la pendiente de subida en dB/oct.}

\subsection{Representar en Excel el patrón polar de directividad de cualquiera de los micrófonos estudiados a la frecuencia de \qty{1}{\kilo \hertz }.}

\subsection{Para el patrón polar anterior (\qty{1}{\kilo \hertz }), obtener en Excel el factor de directividad axial $Q_{ax}$. Usar la expresión de un micrófono con simetría cilíndrica y contrastar el resultado con el valor teórico de la función directividad analizada:}

\begin{equation} \label{eq:qaxial}
  Q_{ax} = \frac{2}{\int_{\theta = 0}^{\pi } D^2 (\theta ) \sen \left( \theta  \right) \dd \theta } \approx \frac{2}{\frac{\pi }{N}\sum_{i=1}^{N}D^2 \left( \theta _i \right)\sen \left( \theta _i \right)}
\end{equation}

\end{document}
