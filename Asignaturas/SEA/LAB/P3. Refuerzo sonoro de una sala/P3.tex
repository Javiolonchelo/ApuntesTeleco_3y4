\documentclass{article}
\usepackage[spanish, es-tabla]{babel}

\usepackage{amsmath}
\usepackage{amssymb}
\usepackage{amsthm}

\usepackage{fontspec}
\usepackage{multicol}
\usepackage{graphicx}
\usepackage{float}
\usepackage[center]{caption}
\usepackage{graphicx}
\usepackage{listings}
\usepackage[dvipsnames, table]{xcolor}
\usepackage{fancyhdr}
\usepackage{titling}
\usepackage[fixlanguage]{babelbib}
\selectbiblanguage{spanish}
\usepackage{subcaption}
\captionsetup{labelfont=bf}
% \captionsetup[table]{labelfont={bf},name={Tabla},labelsep=period}
\usepackage{hyperref}
\definecolor{blueX}{HTML}{3f84e4}
\setmonofont{SFMonoRegular.otf}
\usepackage[margin=2.8cm]{geometry}
\hypersetup{
    colorlinks,
    citecolor=black,
    filecolor=black,
    linkcolor=black,
    urlcolor=black
}
\usepackage{siunitx}
\sisetup{
    per-mode = fraction,
    detect-all,
    exponent-product = \cdot
}
\usepackage{enumitem}
\usepackage{pdflscape}
\usepackage[stylemods,style=super, nogroupskip, toc=false, hyperfirst=false, nonumberlist]{glossaries-extra}

\setlength{\parindent}{0pt} % To avoid indentation
\setabbreviationstyle[acronym]{long-short}
\makenoidxglossaries
\newglossary{difusion}{difusionin}{difusionout}{Glosario}
\loadglsentries{glosario}

% COLUMNAS Y FILAS MULTIPLES DENTRO DE TABLAS
\usepackage{multirow, hhline}
\usepackage{array}
\newcolumntype{L}[1]{>{\raggedright\let\newline\\\arraybackslash\hspace{0pt}}m{#1}}
\newcolumntype{C}[1]{>{\centering\let\newline\\\arraybackslash\hspace{0pt}}m{#1}}
\newcolumntype{R}[1]{>{\raggedleft\let\newline\\\arraybackslash\hspace{0pt}}m{#1}}
\renewcommand{\listfigurename}{Figuras}
\renewcommand{\listtablename}{Tablas}

% \renewcommand{\familydefault}{\sfdefault}
\renewcommand{\thesubsubsection}{\alph{subsubsection}}

\usepackage{titlesec}
\titleformat*{\subsection}{\normalsize\bfseries}
\titleformat*{\subsubsection}{\normalsize\bfseries}

\titlespacing\subsection{0pt}{12pt plus 4pt minus 2pt}{0pt plus 2pt minus 2pt}
\titlespacing\subsubsection{0pt}{12pt plus 4pt minus 2pt}{0pt plus 2pt minus 2pt}

\begin{document}

\title{\textbf{Práctica 3\\Refuerzo sonoro de una sala}}
\author{Javier Rodrigo López}
\date{\today}
\maketitle

\pagenumbering{gobble}

\fancypagestyle{firststyle}
{
    \fancyhead[L]{Sistemas Electroacústicos}
    \fancyhead[R]{\includegraphics[width=0.3\linewidth]{Imágenes/ETSIST.png}}
    %\fancyfoot[R]{\includegraphics[width=0.08\linewidth]{Imágenes/Plantilla_IAC.png}\\Departamento de Ingeniería Audiovisual y Comunicaciones}
}

\thispagestyle{firststyle}

% \newpage

\fancyhead{}
\pagestyle{fancy}

\pagestyle{fancy}
\fancyhead[L]{Difusión de Contenidos Audiovisuales}
\fancyhead[R]{Curso 2023/24}

%\newpage
%\tableofcontents
%\listoffigures
\setcounter{figure}{0}
\setlength{\parskip}{0.5em}

\hypersetup{
    citecolor=black,
    filecolor=black,
    linkcolor=black,
    urlcolor=blueX
}

\tableofcontents
\listoffigures

% IMPRIME ACRÓNIMOS, SIGLAS Y GLOSARIO
%\glsaddall
%\newpage
%\printnoidxglossaries

\newpage

\pagenumbering{arabic}
\setcounter{page}{2}

\section{Modelos de altavoz}
\section{Colocación de los altavoces}
\section{Ajuste de la uniformidad de campo directo a 1 kHz}
\section{Balance de potencia}
\section{Ecualización}
\section{Niveles promedio de banda ancha con y sin ponderación A}
\section{Retardos}
\section{Inteligibilidad}
\section{Cálculo de la respuesta temporal en los puntos 1 y 3}
\section{Auralización}

%\nocite{*}
%\newpage
%\bibliography{Bibliography}
%\bibliographystyle{babplain}

\end{document}
