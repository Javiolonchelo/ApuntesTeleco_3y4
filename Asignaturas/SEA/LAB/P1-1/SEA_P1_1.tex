\documentclass[10pt]{article}



\usepackage{amsmath}
\usepackage{amssymb}
\usepackage{amsthm}
\usepackage{array}
\usepackage{babelbib}
\usepackage{braket}
\usepackage{caption}
\usepackage{colortbl}
\usepackage{rotating}
\usepackage[table]{xcolor}
\usepackage{color}
\usepackage{enumerate}
\usepackage{esint}
\usepackage{eso-pic}
\usepackage{listings}
\usepackage{lscape}
\usepackage{mathtools}
\usepackage{multicol}
\usepackage{multirow}
\usepackage{siunitx}
\usepackage{subcaption}
\usepackage{subdepth}
\usepackage{tcolorbox}
\usepackage{tikz}
\usepackage{titlesec}
\usepackage{titling}
\usepackage{upgreek}
\usepackage{url}
\usepackage{verbatim}
\usepackage{vwcol}
\usepackage{wallpaper}
\usepackage{xfrac}
\usepackage{physics}
\usepackage[c]{esvect}
\usepackage[utf8]{inputenc}
\usepackage[fleqn]{nccmath}
\usepackage[thicklines]{cancel}
\usepackage[margin=2cm]{geometry}
\usepackage[colorlinks=true,spanish]{hyperref}
\usepackage[oldvoltagedirection]{circuitikz}
\usepackage[greek,spanish,es-tabla,es-nodecimaldot,es-noindentfirst]{babel}
\usepackage[symbol]{footmisc}
\renewcommand{\thefootnote}{\fnsymbol{footnote}}

\sisetup{
  per-mode = fraction,
  detect-all,
  exponent-product = \cdot
}
\hypersetup{
  citecolor = blue,
  linkcolor = blue,
  urlcolor = blue,
  pdfauthor = {Javier Rodrigo López}
}
\captionsetup[figure]{labelfont={bf},name={Figura},labelsep=period}
\captionsetup[table]{labelfont={bf},name={Tabla},labelsep=period}
\titleformat{\section}{\normalfont\Large\bfseries}{\thesection}{1em}{}[{\titlerule[0.8pt]}]
\titleformat{\subsubsection}{\normalfont\normalsize\bfseries}{\thesubsubsection}{1em}{}[{}]
\titlespacing{\section}{0pt}{2\parskip}{\parskip}
\titlespacing{\subsection}{0pt}{\parskip}{0pt}
\titlespacing{\subsubsection}{0pt}{\parskip}{0pt}
\usepackage{enumitem}
\setlist{before={\parskip=3pt}, after=\vspace{\baselineskip}}
\setlength{\parindent}{0pt}
\setlength{\parskip}{0.5em}
\renewcommand\thesubsubsection{\arabic{subsubsection}}

\usepackage{booktabs}
\usepackage{bigstrut}

\renewcommand{\vec}{\vv}

% Tipografía
\renewcommand{\familydefault}{\sfdefault}
\renewcommand{\rmdefault}{\sfdefault}

% Para escribir decibelios SPL
\DeclareSIUnit\dbspl{dB\ensuremath{_{\textnormal{SPL}}}}
\DeclareSIUnit\dBlin{dB\ensuremath{_{\textnormal{Lin}}}}
\DeclareSIUnit\dBA{dB\ensuremath{_{\textnormal{A}}}}


\title{\Huge Práctica 1.1. Altavoces \\\huge Laboratorio de Sistemas Electracústicos}
\author{Javier Rodrigo López}
\date{\today}

\begin{document}
\maketitle
% \tableofcontents

\subsubsection{Usando el cursor de la gráfica de función de transferencia $H_1$ del \textit{woofer}, obtener su valor de módulo y fase a la frecuencia de \SI{500}{\Hz}. Proporcionar dicho módulo $\abs{H_1}$ en \SI{}{\dB} re. \SI{20}{\micro\pascal\per\volt} y en \SI{}{\dB} ref. \SI{1}{\pascal\per\volt} usando las propiedades de la gráfica. Comprobar la equivalencia de ambos valores.}

Posicionando el cursor en la gráfica de la función de transferencia a la frecuencia de \SI{500}{\Hz}, leemos el módulo y la fase de esta. Para cambiar la presión de referencia, se hace clic derecho en la gráfica, se selecciona la opción \textit{Properties} y, en la pestaña \textit{Functions}, se cambia el valor de \textit{DB Reference} de \verb|20.0000u| a \verb|1.0000|. Los valores leídos son los siguientes:
\begin{align*}
  \abs{H_1}   & = \SI{82.169}{\dB} \text{ re. } \SI{20}{\micro\pascal\per\volt} & \abs{H_1}   & = \SI{-11.810}{\dB} \text{ re. } \SI{1}{\pascal\per\volt} \\
  \angle{H_1} & = \SI{19.146}{\degree}                                          & \angle{H_1} & = \SI{19.146}{\degree}
\end{align*}


\subsubsection{A partir de los resultados del laboratorio y tomando el valor de $H_1$ del apartado anterior, obtener la sensibilidad del \textit{woofer} usando estas tres expresiones:}
\begin{align}
  S \left[ \SI{}{\dB\per\watt} \right] & = SPL_{\Delta} (r) + 20 \log \left( \frac{2.83}{V_{\Delta}} \right) + 20 \log \left( r \right)                                                                                                                       \\ \label{eq:sens_2}
  S \left[ \SI{}{\dB\per\watt} \right] & = H_1(r) \left[ \SI{}{\dB}\text{ re. }\SI{20}{\micro\pascal\per\volt} \right] + 20 \log \left( \frac{2.83}{1} \right) + 20 \log \left( r \right)                                                                     \\ \label{eq:sens_3}
  S \left[ \SI{}{\dB\per\watt} \right] & = H_1(r) \left[ \SI{}{\dB}\text{ re. }\SI{1}{\pascal\per\volt} \right] + 20 \log \left( \frac{2.83}{1} \right) + 20 \log \left( r \right) - \underbrace{20 \log \left( p _{\textnormal{ref}} \right)}_{\SI{94}{\dB}}
\end{align}

En primer lugar, se debe establecer el rango de frecuencias útil del altavoz a partir de la gráfica de la función de transferencia. En este caso se ha escogido que este rango sea entre \SI{126.5}{\Hz} y \SI{2.563}{\kilo\Hz}.

\subsubsection{Calcular en Excel el valor $SPL_{\Delta}$ mediante las gráficas de la práctica.}

\subsubsection{Mediante la fase de $H_1$ a \SI{600}{\Hz}, estimar la distancia $r$ a la que estaba el micrófono. Suponer que a esa frecuencia el altavoz tiene fase nula y la única fase captada es la fase acústica, $\varphi _a = -kr = \frac{-2\pi fr}{c}$.}

\end{document}
