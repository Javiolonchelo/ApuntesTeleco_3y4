\documentclass[10pt]{article}



\usepackage{amsmath}
\usepackage{amssymb}
\usepackage{amsthm}
\usepackage{array}
\usepackage{babelbib}
\usepackage{braket}
\usepackage{caption}
\usepackage{colortbl}
\usepackage{rotating}
\usepackage[table]{xcolor}
\usepackage{color}
\usepackage{enumerate}
\usepackage{esint}
\usepackage{eso-pic}
\usepackage{listings}
\usepackage{lscape}
\usepackage{mathtools}
\usepackage{multicol}
\usepackage{multirow}
\usepackage{siunitx}
\usepackage{subcaption}
\usepackage{subdepth}
\usepackage{tcolorbox}
\usepackage{tikz}
\usepackage{titlesec}
\usepackage{titling}
\usepackage{upgreek}
\usepackage{url}
\usepackage{verbatim}
\usepackage{vwcol}
\usepackage{wallpaper}
\usepackage{xfrac}
\usepackage{physics}
\AtBeginDocument{\RenewCommandCopy\qty\SI}
\usepackage[c]{esvect}
\usepackage[utf8]{inputenc}
\usepackage[fleqn]{nccmath}
\usepackage[thicklines]{cancel}
\usepackage[margin=2cm]{geometry}
\usepackage[colorlinks=true,spanish]{hyperref}
\usepackage[oldvoltagedirection]{circuitikz}
\usepackage[greek,spanish,es-tabla,es-nodecimaldot,es-noindentfirst]{babel}
\usepackage[symbol]{footmisc}
\renewcommand{\thefootnote}{\fnsymbol{footnote}}

\sisetup{
  per-mode = fraction,
  inline-per-mode = power,
  detect-all,
  exponent-product = \cdot
}
\hypersetup{
  citecolor = blue,
  linkcolor = blue,
  urlcolor = blue,
  pdfauthor = {Javier Rodrigo López}
}
\captionsetup[figure]{labelfont={bf},name={Figura},labelsep=period}
\captionsetup[table]{labelfont={bf},name={Tabla},labelsep=period}
\titleformat{\section}{\normalfont\Large\bfseries}{\thesection}{1em}{}[{\titlerule[0.8pt]}]
\titleformat{\subsubsection}{\normalfont\normalsize\bfseries}{\thesubsubsection}{1em}{}[{}]
\titlespacing{\section}{0pt}{2\parskip}{\parskip}
\titlespacing{\subsection}{0pt}{\parskip}{0pt}
\titlespacing{\subsubsection}{0pt}{\parskip}{0pt}
\usepackage{enumitem}
\setlist{before={\parskip=3pt}, after=\vspace{\baselineskip}}
\setlength{\parindent}{0pt}
\setlength{\parskip}{0.5em}
\renewcommand\thesubsubsection{\arabic{subsubsection}}

\usepackage{booktabs}
\usepackage{bigstrut}

\renewcommand{\vec}{\vv}

% Tipografía
% \renewcommand{\familydefault}{\sfdefault}
% \renewcommand{\rmdefault}{\sfdefault}

% Para escribir decibelios SPL
\DeclareSIUnit\dbspl{dB\ensuremath{_{\textnormal{SPL}}}}
\DeclareSIUnit\dBlin{dB\ensuremath{_{\textnormal{Lin}}}}
\DeclareSIUnit\dBA{dB\ensuremath{_{\textnormal{A}}}}


\title{\Huge Práctica 1.2. Altavoces \\\huge Laboratorio de Sistemas Electroacústicos}
\author{Javier Rodrigo López}
\date{\today}

\begin{document}
\maketitle
% \tableofcontents

\subsubsection{Representar en Excel la respuesta anecoica del \textit{woofer} componiéndola a partir de las medidas de campo lejano y cercano alrededor de $f=\qty{300}{\hertz }$. Los datos de campo cercano deben atenuarse para alcanzar la curva de campo lejano. Utilizar unidades de \unit{\dB\per\volt} y una escala logarítmica de frecuencias desde \qty{20}{\hertz}.}

\subsubsection{Adjuntar las respuestas medidas del sistema de altavoces de tres vías con los filtros de cruce
  (polaridad correcta, inversión de polaridad del midrange, desplazamiento del tweeter...)
  Explicar el porqué de las diferencias entre curvas y por qué esas diferencias se producen sólo a
  determinadas frecuencias.}

\subsubsection{Adjuntar las funciones de transferencia módulo/fase de los filtros de cruce (caso de carga
  8 Ω). Adjuntar en forma de tabla:
  • las frecuencias de cruce fc1 y fc2,
  • las ganancias en fc1 y fc2 de los filtros que se cruzan,
  • la fase relativa (o diferencia de fase) de las vías implicadas en cada frecuencia de cruce y
  • las pendientes de ganancia/atenuación2 de los filtros en dB/oct y dB/dec.}

\subsubsection{Usando las curvas de impedancia eléctrica de entrada para el caso «woofer en caja
  hermética»3,calcular por el método de Small fc, Qec, Qmc, Qtc, α (relación de compliancias del
  sistema de caja acústica cerrada) y Vab (volumen acústico equivalente). Usar los datos del
  fabricante (página 110 del libro). Obtener también γ, la constante termodinámica de la caja.
  Dato4: Vb = 0.158 m3.}

\end{document}
