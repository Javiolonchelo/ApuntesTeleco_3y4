\documentclass[a4paper, 8pt]{extarticle}

\usepackage[greek,spanish,es-tabla,es-nodecimaldot,es-noindentfirst]{babel}

\usepackage[a4paper, lmargin=0.2cm,rmargin=0.2cm,tmargin=1cm,bmargin=1cm, landscape]{geometry}
\usepackage{multicol}
\usepackage{amsmath}
\usepackage{mathtools}
\usepackage{cancel}
\usepackage{siunitx}
\usepackage{physics}
\usepackage{enumitem}
\usepackage{circuitikz}

% For Excel2LaTeX tables
\usepackage{xcolor}
\usepackage{colortbl}
\usepackage{booktabs}

\AtBeginDocument{\RenewCommandCopy\qty\SI}
\usepackage{esvect}
\renewcommand{\vec}[1]{\vv{{#1}}}
\renewcommand{\grad}{\nabla}

\usepackage{lmodern}
\renewcommand{\familydefault}{\sfdefault}
\renewcommand{\rmdefault}{\sfdefault}

\renewcommand{\sin}{\sen}

\usepackage{titlesec}
\titleformat{\section}
  {\normalfont\Large\bfseries}{\thesection}{1em}{}[{\titlerule[0.8pt]}]

\titlespacing*{\section}{0pt}{4pt plus 0pt minus 0pt}{5pt plus 2pt minus 2pt}
\titlespacing*{\subsection}{0pt}{4pt plus 0pt minus 0pt}{3pt plus 2pt minus 2pt}
\titlespacing*{\subsubsection}{0pt}{4pt plus 0pt minus 0pt}{3pt plus 2pt minus 2pt}

\allowdisplaybreaks
\setcounter{secnumdepth}{-1}
\setcounter{tocdepth}{-1}

%%% INICIO DEL DOCUMENTO %%%
\begin{document}

\setlength{\parskip}{0pt}
\setlength{\parindent}{0pt}
\setlist{nosep}
\setlength{\abovedisplayskip}{2pt}
\setlength{\belowdisplayskip}{2pt}
\setlength{\abovedisplayshortskip}{0pt}
\setlength{\belowdisplayshortskip}{0pt}


\pagestyle{empty}
\renewcommand{\arraystretch}{1.5}

\begin{multicols}{3}
  \section{T3. Micrófonos}
  \subsection{Características de los micrófonos}
  \subsubsection{Sensibilidad}
  La sensibilidad de un micrófono es la \textbf{relación entre la tensión eléctrica generada y la presión acústica incidente} (en circuito abierto, campo libre, a \qty{1}{\kilo\hertz}).
  \begin{align*}
    S & = \frac{E_{\text{c.a.}}}{p} \ \left[ \unit{\volt \per \pascal }  \right]                                                                            \\
      & = 20 \log \left( \frac{\abs{S}}{\qty{1}{\volt \per \pascal  } } \right) \ \left[ \unit{\decibel} \text{ re. } \qty{1}{\volt \per \pascal }  \right]
  \end{align*}
  La sensibilidad de referencia es la sensibilidad medida en el eje del micrófono.
  \[ S_0 = S \left( \theta = \ang{0} \right) \]
  \subsubsection{Respuesta en frecuencia}
  La respuesta en frecuencia de un micrófono es la \textbf{variación de la sensibilidad con la frecuencia} $\left[ S = S(f) \right]$. Se suele representar la respuesta relativa con respecto a la sensibilidad de referencia en decibelios.
  \[ S(f) - S_0 = 20 \log \left( \frac{\abs{S(f)}}{\abs{S_0}} \right) \ \unit{\dB} \]
  La presencia del micrófono afecta a su respuesta. La respuesta del diafragma en alta frecuencia aumenta la presión frente a la cápsula.
  \subsubsection{Distorsión lineal}
  \begin{itemize}
    \item \textbf{Coloraciones.} La respuesta en frecuencia no es plana.
    \item \textbf{Vibraciones parciales del diafragma.} Debidas a modos propios. Los micros de condensador no tienen este problema.
    \item \textbf{Resonancias mecánicas o acústicas.} Sobre todo en micros dinámicos.
    \item \textbf{Ancho de banda limitado por componentes eléctricos.}
    \item \textbf{Distorsión de fase.}
  \end{itemize}
  \subsubsection{Distorsión no lineal}
  \begin{itemize}
    \item \textbf{Saturación.} Sobrecarga de presión. Si saturan fácilmente se llaman micros \textit{blandos}. Si no, \textit{duros}.
    \item \textbf{Pop.} Chorros de aire (no sonido). Suele ser por fonemas ``explosivos''.
  \end{itemize}
  \subsubsection{Directividad}
  Generalmente, su supone simetría cilíndrica de los micrófonos, por lo que la directividad no depende del plano azimutal $\varphi$.
  \begin{align*}
    D(\theta, \varphi )          & \equiv \text{Directividad}                                         \\
    Q(\theta, \varphi )          & \equiv \text{Factor de directividad}                               \\
    Q _{\textnormal{ax}}         & \equiv \text{Factor de directividad axial (en el eje)}             \\
    \text{DI}                    & \equiv \text{Índice de directividad}                               \\
    \text{DI} _{\textnormal{ax}} & \equiv \text{Índice de directividad axial (en el eje)}             \\
    \text{REE}                   & \equiv \text{Eficiencia de energía aleatoria}                      \\
    \text{DSF}                   & \equiv \text{Factor de distancia}                                  \\
    E_d                          & \equiv \text{Tensión directa}                                      \\
    E_r                          & \equiv \text{Tensión reverberante}                                 \\
    E_{ro}                       & \equiv \text{Tensión reverberante del equivalente omnidireccional} \\
  \end{align*}
  Si especifican que se ha medido en cámara anecoica, entonces se refiere a que $E_{\text{c.a.}}$ es también el valor de $E_d$.
  \[ E_r  = \frac{p_r S_0}{\sqrt{Q _{\textnormal{ax}}}} \qquad  E_{ro} = p_r S \]
  \textbf{Nota 1:} En la fórmula de $Q_{\textnormal{ax}}$ para $N$ valores, si nos dicen que supongamos simetría de revolución se usan los valores entre $\theta = \ang{0}$ y $\theta = \ang{180}$, sin incluir este último. Si da tiempo (y en entornos reales), se calculan tanto en el intervalo $[0, 180)$ como en $[180, 360)$ y se hace la media entre ambos $Q_{\textnormal{ax}}$ obtenidos.

  \textbf{Nota 2:} Si nos piden sacar la directividad sabiendo que es de familia cardioide y no dicen nada más, debemos suponer orden $n=1$.
  \begin{align*}
    D( \theta, \varphi ) & = \frac{S(\theta, \varphi )}{S_0 }                                                                                                                                                                                                \\
    Q(\theta, \varphi )  & = \frac{S^2(\theta, \varphi )}{\langle S^2(\theta, \varphi) \rangle} = \frac{S^2(\theta, \varphi )}{\frac{1}{4\pi} \int_{\varphi = 0}^{2\pi} \int_{\theta = 0}^{\pi} S^2(\theta, \varphi) \sin (\theta ) \dd \theta \dd \varphi } \\
    Q_{\textnormal{ax}}  & = Q(\theta = \ang{0}) = \frac{2}{\int_{\theta = 0}^{\pi} D^2(\theta) \sin (\theta ) \dd \theta } \approx \frac{2}{\frac{\pi}{N} \sum _{i=1}^{N} D^2(\theta _i) \sin (\theta i)}                                                   \\
    Q_{\textnormal{ax}}  & = \frac{E^2_{ro}} {E^2_r} \eval*{}_{\text{campo difuso}} \qquad \text{DI} = 10 \log \left[ Q( \theta, \varphi) \right] \qquad \text{DI} _{\textnormal{ax}} = 10 \log \left( Q_{\textnormal{ax}} \right)                           \\
    \text{REE}           & = \frac{1}{Q_ {\textnormal{ax}}} \qquad \text{DSF} = \sqrt{Q_{\textnormal{ax}}}                                                                                                                                                   \\
  \end{align*}
  \subsubsection{Directividad de la familia cardiode}
  $A$ es el componente omnidireccional, $B$ el componente bidireccional y $n$ el orden de la directividad.

  \[\left\lbrace
    \begin{matrix*}[l]
      D (\theta ) = \left[ A + B \cos (\theta ) \right] \cos ^{n-1} (\theta )  \\
      A + B = 1
    \end{matrix*}\right.\]


  % Table generated by Excel2LaTeX from sheet 'Sheet1'
  \begin{center}
    \begin{tabular}{|c|c|l|}
      \hline
      \rowcolor[rgb]{ .663,  .816,  .557} A    & B    & \multicolumn{1}{c|}{Tipo} \\
      \hline
      \rowcolor[rgb]{ .886,  .937,  .855} 0.50 & 0.50 & Cardioide                 \\
      \hline
      \rowcolor[rgb]{ .886,  .937,  .855} 0.75 & 0.25 & Subcardioide              \\
      \hline
      \rowcolor[rgb]{ .886,  .937,  .855} 0.25 & 0.75 & Hipercardioide            \\
      \hline
    \end{tabular}
  \end{center}

  Si $n = 1$, el valor de $Q _{\textnormal{ax}}$ viene dado por la resolución de la integral de su definición, el resultado es este:

  \[ Q _{\textnormal{ax}} = \frac{3}{4B^2 - 6B + 3} \]


  \subsubsection{Ruido eléctrico}

  \begin{itemize}
    \item \textbf{Ruido eléctrico.} Tensión de salida del micrófono cuando no recibe excitación acústica. Causado por:
          \begin{itemize}
            \item Agitación térmica de moléculas de aire o del diafragma.
            \item Agitación térmica electrónica, debi principalmente a resistencias altas.
          \end{itemize}
          \[ E_N = \sqrt{4kTR \Delta f} \ \left[ \unit{\volt }  \right]\]
          Donde $k$ es la constante de Boltzmann, $T$ la temperatura, $R$ la resistencia y $\Delta f$ el ancho de banda. Se suele expresar en \unit{\dB_{\text{SPL}}} y se denomina ``nivel de presión sonora equivalente al ruido'':
          \[ \text{ENL} = 20 \log \left( \frac{p_N}{p _{\textnormal{ref}}} \right)  = 20 \log \left( \frac{E_N}{p _{\textnormal{ref}}S_0} \right) \]
          Donde $p _{\textnormal{ref}} = \qty{20}{\micro\pascal }$. Para usar esta expresión, mencionar que se debería filtrar $E_N$ con un filtro de ponderación A.
    \item \textbf{Ruido por zumbido electromagnético (\textit{hum}).}
    \item \textbf{Ruido por viento.}
  \end{itemize}

  \subsubsection{Márgenes dinámicos}

  \begin{align*}
    \text{Margen dinámico}        & \qquad \text{DR} = \text{SPL}_{\text{máx}} - \text{ENL}                                    \\
    \text{Margen de sobrecarga}   & \qquad \text{HR} = \text{SPL}_{\text{máx}} - 94 \ (\text{viene de } p _{\textnormal{ref}}) \\
    \text{Relación señal a ruido} & \qquad \text{SNR} = \text{DR} - \text{HR} = 94 - \text{ENL}
  \end{align*}
  \subsubsection{Otras características}

  \subsection{Tipos de micrófonos}
  \subsubsection{TAM: Micrófonos de presión}
  \subsubsection{TAM: Micrófonos de gradiente de presión}
  \subsubsection{TAM: Micrófonos combinados de presión - gradiente de presión}
  \subsubsection{TME: Micrófonos de bobina móvil}
  \subsubsection{TME: Micrófonos de cinta}
  \subsubsection{TME: Micrófonos electrostáticos de condensador}
  \subsubsection{TME: Micrófonos electrostáticos de electret (o prepolarizados)}
  \subsubsection{TME: Micrófonos MEMS}
  \subsubsection{Especiales: Micrófonos de doble diafragma}
  \subsubsection{Especiales: Micrófonos superdirectivos}
  \subsubsection{Especiales: Micrófonos lavalier}
  \subsubsection{Especiales: Microfonía estereofónica}
  \subsubsection{Especiales: Micrófonos de superficie o de zona de presión}
  \subsubsection{Especiales: Micrófonos inalámbricos}

  \subsection{Conexión eléctrica de los micrófonos}
  \subsubsection{Impedancias características}
  \subsubsection{Efecto del cable en la banda de frecuencias transmitida}
  \subsubsection{Línea microfónica balanceada}
  \subsubsection{Alimentación de micrófonos electrostáticos}
  \subsubsection{Adaptadores, conversores y distribuidores microfónicos}

  \newpage
  \section{T4. Sistemas de Refuerzo Sonoro}

  \subsection{Niveles acústicos}
  \subsubsection{Niveles de presión directo y reverberante}
  \subsubsection{Constante acústica de la sala}
  \subsubsection{Modificadores acústicos}
  \subsubsection{Distancia crítica}
  \subsubsection{Campo semirreverberante}
  \subsubsection{Niveles debidos a varias fuentes}
  \subsubsection{Potencia acústica}

  \subsection{Respuesta temporal}
  \subsubsection{Molestia por ecos}
  \subsubsection{Efecto precedencia}
  \subsubsection{Respuesta temporal mediante simulación}
  \subsubsection{Auralización}

  \subsection{Inteligibilidad}
  \subsubsection{Índice de inteligibilidad del habla}
  \subsubsection{Pérdida de articulación de consonantes}
  \subsubsection{Índice de transmisión del habla}

  \subsection{Configuraciones de altavoces}
  \subsubsection{Sistema centralizado con un altavoz o un cluster}
  \subsubsection{Clusters y arrays de altavoces}
  \subsubsection{Sistema distribuidos de altavoces}

  \subsection{Realimentación acústica}
  \subsubsection{Modelo}
  \subsubsection{Realimentación por un solo camino y en campo libre}
  \subsubsection{Realimentación cuando existe reverberación}
  \subsubsection{Condición de oscilación según los niveles acústicos en los micrófonos}
  \subsubsection{Respuesta temporal de la realimentación}
  \subsubsection{Control de la realimentación acústica}

  \subsection{Ganancia acústica}
  \subsubsection{Distancia acústica equivalente}
  \subsubsection{Ganancia acústica necesaria y ganancia acústica potencial}
  \subsubsection{Uso de la ganancia acústica para el diseño de un sistema de refuerzo sonoro}
  \subsection{Amplificación}
  \subsubsection{Amplificación de baja impedancia}
  \subsubsection{Amplificación de alta impedancia, líneas de tensión constante}
  \subsubsection{Conexión de amplificadores}
  \subsubsection{Clases de amplificación}
  \subsubsection{Fuentes de alimentación}

\end{multicols}

\end{document}
