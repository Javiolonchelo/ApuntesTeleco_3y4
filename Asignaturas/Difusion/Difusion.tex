\documentclass[10pt]{book}



\usepackage{amsmath}
\usepackage{amssymb}
\usepackage{amsthm}
\usepackage{array}
\usepackage{babelbib}
\usepackage{braket}
\usepackage{caption}
\usepackage{colortbl}
\usepackage{rotating}
\usepackage[table]{xcolor}
\usepackage{color}
\usepackage{enumerate}
\usepackage{esint}
\usepackage{eso-pic}
\usepackage{listings}
\usepackage{lscape}
\usepackage{mathtools}
\usepackage{multicol}
\usepackage{multirow}
\usepackage{siunitx}
\usepackage{subcaption}
\usepackage{subdepth}
\usepackage{tcolorbox}
\usepackage{tikz}
\usepackage{titlesec}
\usepackage{titling}
\usepackage{upgreek}
\usepackage{url}
\usepackage{verbatim}
\usepackage{vwcol}
\usepackage{wallpaper}
\usepackage{xfrac}
\usepackage{physics}
\usepackage[c]{esvect}
\usepackage[utf8]{inputenc}
\usepackage[fleqn]{nccmath}
\usepackage[thicklines]{cancel}
\usepackage[margin=2cm]{geometry}
\usepackage[colorlinks=true,spanish]{hyperref}
\usepackage[oldvoltagedirection]{circuitikz}
\usepackage[greek,spanish,es-tabla,es-nodecimaldot,es-noindentfirst]{babel}
\usepackage[symbol]{footmisc}
\usepackage{acro}
\renewcommand{\thefootnote}{\fnsymbol{footnote}}

\sisetup{
  per-mode = fraction,
  detect-all,
  exponent-product = \cdot
}
\hypersetup{
  citecolor = blue,
  linkcolor = blue,
  urlcolor = blue,
  pdfauthor = {Javier Rodrigo López}
}
\captionsetup[figure]{labelfont={bf},name={Figura},labelsep=period}
\captionsetup[table]{labelfont={bf},name={Tabla},labelsep=period}
\titleformat{\section}{\normalfont\Large\bfseries}{\thesection}{1em}{}[{\titlerule[0.8pt]}]
\titleformat{\subsubsection}{\normalfont\normalsize\bfseries}{\thesubsubsection}{1em}{}[{\titlerule[0.05pt]}]
\titlespacing{\section}{0pt}{2\parskip}{\parskip}
\titlespacing{\subsection}{0pt}{\parskip}{0pt}
\titlespacing{\subsubsection}{0pt}{\parskip}{0pt}
\usepackage{enumitem}
\setlist{before={\parskip=3pt}, after=\vspace{\baselineskip}}
\setlength{\parindent}{0pt}
\setlength{\parskip}{0.5em}

\usepackage{booktabs}
\usepackage{bigstrut}

\renewcommand{\vec}{\vv}

% Tipografía
\renewcommand{\familydefault}{\sfdefault}
\renewcommand{\rmdefault}{\sfdefault}

% Para escribir decibelios SPL
\DeclareSIUnit\dbspl{dB\ensuremath{_{\textnormal{SPL}}}}
\DeclareSIUnit\dBlin{dB\ensuremath{_{\textnormal{Lin}}}}
\DeclareSIUnit\dBA{dB\ensuremath{_{\textnormal{A}}}}

% Siglas 
\DeclareAcronym{dvb}{
  short=DVB,
  long=\textit{Digital Video Broadcasting},
}
\DeclareAcronym{ts}{
  short=TS,
  long=\textit{Transport Stream},
}
\DeclareAcronym{es}{
  short=ES,
  long=\textit{Elementary Stream},
}
\DeclareAcronym{ps}{
  short=PS,
  long=\textit{Program Stream},
}
\DeclareAcronym{pes}{
  short=PES,
  long=\textit{Packetized Elementary Stream},
}
\DeclareAcronym{pid}{
  short=PID,
  long=\textit{Packet Identifier},
}
\DeclareAcronym{pcr}{
  short=PCR,
  long=\textit{Program Clock Reference},
}
\DeclareAcronym{pmt}
{
  short=PMT,
  long=\textit{Program Map Table},
}
\DeclareAcronym{pat}
{
  short=PAT,
  long=\textit{Program Association Table},
}
\DeclareAcronym{psi}{
  short=PSI,
  long=\textit{Program Specific Information}, 
}
\DeclareAcronym{dvbsi}{
  short=DVB-SI,
  long=\textit{Service Information},
}

\title{\Huge Difusión de contenidos audiovisuales\\\huge Apuntes de clase}
\author{Javier Rodrigo López}
\date{\today}

\begin{document}

% Este comando crea el título
\maketitle

% Este comando crea el índice
\tableofcontents

\printacronyms
\acresetall


\newpage

\section*{Presentación}

Nota mínima para laboratorio: 4

Nota mínima para teoría: 5

Nota final: Teoría 70\% (Ambas partes hay que aprobarlas y cuentan un 35\%), Laboratorio 30\%

Examen de laboratorio es un 50\% de la nota de laboratorio.

El laboratorio liberado se guarda para cursos posteriores.

Intentar aprobar los parciales para no hacer examen en enero !!!!!

\newpage



\chapter{Cabeceras de difusión de contenidos audiovisuales}
\section{Organización de los contenidos a emitir: soportes proveedores, \mbox{HDTV/SDTV}}
La organización de los datos se realiza con la capa de sistema de MPEG-2. A través de un solo canal pueden distribuirse varios programas, al menos en televisión digital. MPEG-4 puede llegar a comprimir de \SI{1.5}{Gbps} a \SI{4}{Mbps}.

\begin{itemize}
  \item \textbf{Proveedor de servicios}.
  \item \textbf{Operador de televisión}. Es común que el operador de televisión sea el mismo que el proveedor de servicios. Hoy en día, este elemento se encarga de montar un múltiplex, que en MPEG-2 se denomina \ac{ts}, con los programas generados por los proveedores de servicios, que entrega finalmente al operador de transmisión.
  \item \textbf{Operador de transmisión}.
\end{itemize}

\subsection{Normativa de transmisión}

\begin{itemize}
  \item Radiodifusión terrestre. Se guía por la norma ETS 300 744. Se utilizan canales de \SI{7} u \SI{8}{\mega\Hz}.
  \item Radiodifusión por satélite. Se guía por la norma ETS 300 421. Se trabaja en la banda de \SI{11} o \SI{12}{\giga\Hz}.
  \item Transmisión por cable. Se guía por la norma ETS 300 429. Se trabaja en la banda de \SI{8}{\mega\Hz}.
\end{itemize}

\subsection{Descripción de la capa de sistema de MPEG-2}

Cada flujo de contenidos que sale de la campa de compresión se denomina \ac{es} y es un grupo de datos continuo que puede ser de vídeo, de audio, etc.

Cada \ac{es} puede ser empaquetado en diferentes paquetes denominados \ac{pes}.

Estos nuevos elementos, que tendrán un \ac{pid}, se pueden empaquetar en un \ac{ps} o bien en un \ac{ts}. Los \ac{ts} son secuencias de \ac{ps} con diferentes orígenes con un tamaño fijo de \SI{188}{bytes}. Contienen información de sistema, información de programa y contenidos.

Cada programa tiene una \ac{pmt} y en ella se hallan los \ac{pid} de los \ac{es} que lo componen.

Según el medio de distribución, se realiza la demodulación \ac{dvb} que corresponda, y entonces se procesa el \ac{ts} para obtener los \ac{pes} y los paquetes de sistema, y entonces se procesan los \ac{pes} para obtener los \ac{es}.
\section{Necesidades según destinatarios}

\chapter{Encapsulado de vídeo y audio para difusión}
\section{Capa de sistema MPEG: Transport Stream}
\section{Tablas Service Information (SI)}
\section{Temporización}


\chapter{Servicios añadidos, interactividad y acceso condicional}

\section{Teletexto}

El teletexto siempre ha sido digital. Sin embargo, al principio se emitía sobre la señal de televisión analógica.

\section{EPG}
\section{Acceso Condicional}
\section{Interactividad - HbbTV}

\chapter{Difusión y distribución audiovisual en redes de televisión}
\section{Difusión digital DVB: cable, satélite, terrestre y portable}
\section{Distribución profesional por satélite: DVB-DSNG}
\section{Sistemas de recepción de televisión: profesional e ICT}

\chapter{Difusión y distribución audiovisual por redes genéricas de datos}
\section{Streaming en la web}
\section{Videoconferencia sobre IP}
\section{IPTV}

\chapter{Difusión y distribución audiovisual sobre soportes autónomos}
\section{Autoría}
\section{Gestión de derechos digitales (DRM)}

\end{document}
