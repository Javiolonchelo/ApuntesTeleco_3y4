\documentclass[10pt]{article}



\usepackage{amsmath}
\usepackage{amssymb}
\usepackage{amsthm}
\usepackage{array}
\usepackage{babelbib}
\usepackage{braket}
\usepackage{caption}
\usepackage{colortbl}
\usepackage{rotating}
\usepackage[table]{xcolor}
\usepackage{color}
\usepackage{enumerate}
\usepackage{esint}
\usepackage{eso-pic}
\usepackage{listings}
\usepackage{lscape}
\usepackage{mathtools}
\usepackage{multicol}
\usepackage{multirow}
\usepackage{siunitx}
\usepackage{subcaption}
\usepackage{subdepth}
\usepackage{tcolorbox}
\usepackage{tikz}
\usepackage{titlesec}
\usepackage{titling}
\usepackage{upgreek}
\usepackage{url}
\usepackage{verbatim}
\usepackage{vwcol}
\usepackage{wallpaper}
\usepackage{xfrac}
\usepackage{physics}
\usepackage[c]{esvect}
\usepackage[utf8]{inputenc}
\usepackage[fleqn]{nccmath}
\usepackage[thicklines]{cancel}
\usepackage[margin=2cm]{geometry}
\usepackage[colorlinks=true,spanish]{hyperref}
\usepackage[oldvoltagedirection]{circuitikz}
\usepackage[greek,spanish,es-tabla,es-nodecimaldot,es-noindentfirst]{babel}
\usepackage[symbol]{footmisc}
\renewcommand{\thefootnote}{\fnsymbol{footnote}}

\sisetup{
  per-mode = fraction,
  detect-all,
  exponent-product = \cdot
}
\hypersetup{
  citecolor = blue,
  linkcolor = blue,
  urlcolor = blue,
  pdfauthor = {Javier Rodrigo López}
}
\captionsetup[figure]{labelfont={bf},name={Figura},labelsep=period}
\captionsetup[table]{labelfont={bf},name={Tabla},labelsep=period}
\titleformat{\section}{\normalfont\Large\bfseries}{\thesection}{1em}{}[{\titlerule[0.8pt]}]
\titleformat{\subsubsection}{\normalfont\normalsize\bfseries}{\thesubsubsection}{1em}{}[{\titlerule[0.05pt]}]
\titlespacing{\section}{0pt}{2\parskip}{\parskip}
\titlespacing{\subsection}{0pt}{\parskip}{0pt}
\titlespacing{\subsubsection}{0pt}{\parskip}{0pt}
\usepackage{enumitem}
\setlist{before={\parskip=3pt}, after=\vspace{\baselineskip}}
\setlength{\parindent}{0pt}
\setlength{\parskip}{0.5em}

\usepackage{booktabs}
\usepackage{bigstrut}

% Tipografía
\renewcommand{\familydefault}{\sfdefault}
\renewcommand{\rmdefault}{\sfdefault}

% Para escribir decibelios SPL
\DeclareSIUnit\dbspl{dB\ensuremath{_{\textnormal{SPL}}}}
\DeclareSIUnit\dBlin{dB\ensuremath{_{\textnormal{Lin}}}}
\DeclareSIUnit\dBA{dB\ensuremath{_{\textnormal{A}}}}
 % Se incluye el preámbulo

\title{\Huge Sistemas basados en microprocesador\\\huge Apuntes de clase}
\author{Javier Rodrigo López}
\date{\today}

\begin{document}

% Este comando crea el título
\maketitle

% Este comando crea el índice
\tableofcontents


\newpage

\section{Presentación}

\subsection{Datos descrptivos}

\subsection{Profesorado}

\begin{itemize}
  \item Julián Nieto Valhondo (Coordinador) \href{julian.nieto.valhondo@upm.es}{mailto:julian.nieto.valhondo@upm.es}
  \item Eduardo Barrera López de Turiso \href{eduardo.barrera@upm.es}{mailto:eduardo.barrera@upm.es}
  \item Mariano Ruiz González \href{mariano.ruiz@upm.es}{mailto:mariano.ruiz@upm.es}
\end{itemize}

\subsection{Contenidos}

\begin{itemize}
  \item Bloque 1. Microcontroladores Cortex-M de ARM. Keil ARM CSIS
        \begin{itemize}
          \item Arquitectura interna. Elementos fucnionales, reloj, interrupciones.
          \item Introducción al desarrollo de aplicaciones para microcontroladores Cortex M con CMSIS.
          \item GPIOs-Timers
        \end{itemize}
  \item Bloque 2. Bloques funcionales de un sistema basado en microprocesador
        \begin{itemize}
          \item CMSIS Driver - Interfaces de usuario
          \item Sensores / Actuadores
          \item Comunicaciones
          \item Técnicas de desarrollo de aplicaciones - RTOS
        \end{itemize}
  \item Bloque 3. Integración y desarrollo de una aplicación
\end{itemize}

\subsection{Calendario}

\subsection{Metodología}

Clase de laboratorio: A3002
Por parejas

\subsection{Evaluación}

\subsubsection{Convocatoria ordinaria}
\textbf{Test y examen práctico de los Bloques 1 y 2.} Son dos pruebas separadas, evaluadas por hitos. Las partes que se superen a lo largo del semestre, se guardan para no hacerla en las siguientes pruebas. El primer bloque tiene un peso del 15\%, el segundo bloque tiene un 20\%.

Es obligatoria la asistencia al laboratorio, al igual que la entrega de las prácticas en tiempo y forma.

Para todas las pruebas, tendremos disponible toda la documentación.

\textbf{Calificación del Bloque 3.} Examen: Que el trabajo funcione según las especificaciones, que contestemos bien a las preguntas y que la memoria técnica sea adecuada.

Todas las pruebas tienen que superar una nota mínima de 4/10.

\subsubsection{Convocatoria extraordinaria}
Se compondrá por test y examen práctico de los Bloques 1 y 2, además de un diseño de mediana complejidad para la evaluación del Bloque 3.

\subsection{Recursos}

\begin{itemize}
  \item Préstamo de la tarjeta \verb|STM32-Nucleo-F429ZI|
  \item Préstamo de la \verb|mbed Application board|
  \item Cable USB a micro-USB
  \item Cables estilo Arduino macho-macho.
\end{itemize}

\section{Bloque 1}

\subsection{CMSIS Core}

En \verb|driver.c| se define la API para cada uno de los periféricos que integra el microcontrolador.


\end{document}
