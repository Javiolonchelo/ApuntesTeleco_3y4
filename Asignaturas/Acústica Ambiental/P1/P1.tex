\documentclass[11pt]{article}
\usepackage[spanish, es-tabla]{babel}

\usepackage{amsmath}
\usepackage{amssymb}
\usepackage{amsthm}

\usepackage{fontspec}
\usepackage{multicol}
\usepackage{graphicx}
\usepackage{float}
\usepackage[center]{caption}
\usepackage{graphicx}
\usepackage{listings}
\usepackage[dvipsnames, table]{xcolor}
\usepackage{fancyhdr}
\usepackage{titling}
\usepackage[fixlanguage]{babelbib}
\selectbiblanguage{spanish}
\usepackage{subcaption}
\captionsetup{labelfont=bf}
% \captionsetup[table]{labelfont={bf},name={Tabla},labelsep=period}
\usepackage{hyperref}
\definecolor{blueX}{HTML}{3f84e4}
\setmonofont{SFMonoRegular.otf}
\usepackage[margin=2.8cm]{geometry}
\hypersetup{
    colorlinks,
    citecolor=black,
    filecolor=black,
    linkcolor=black,
    urlcolor=black
}
\usepackage{siunitx}
\sisetup{
    per-mode = fraction,
    detect-all,
    exponent-product = \cdot
}
\usepackage{enumitem}
\usepackage{pdflscape}
\usepackage[stylemods,style=super, nogroupskip, toc=false, hyperfirst=false, nonumberlist]{glossaries-extra}

\setlength{\parindent}{0pt} % To avoid indentation
\setabbreviationstyle[acronym]{long-short}
\makenoidxglossaries
\newglossary{difusion}{difusionin}{difusionout}{Glosario}
\loadglsentries{glosario}

% COLUMNAS Y FILAS MULTIPLES DENTRO DE TABLAS
\usepackage{multirow, hhline}
\usepackage{array}
\newcolumntype{L}[1]{>{\raggedright\let\newline\\\arraybackslash\hspace{0pt}}m{#1}}
\newcolumntype{C}[1]{>{\centering\let\newline\\\arraybackslash\hspace{0pt}}m{#1}}
\newcolumntype{R}[1]{>{\raggedleft\let\newline\\\arraybackslash\hspace{0pt}}m{#1}}
\renewcommand{\listfigurename}{Figuras}
\renewcommand{\listtablename}{Tablas}

% \renewcommand{\familydefault}{\sfdefault}
\renewcommand{\thesubsubsection}{\alph{subsubsection}}

\usepackage{titlesec}
\titleformat*{\subsection}{\normalsize\bfseries}
\titleformat*{\subsubsection}{\normalsize\bfseries}

\titlespacing\subsection{0pt}{12pt plus 4pt minus 2pt}{0pt plus 2pt minus 2pt}
\titlespacing\subsubsection{0pt}{12pt plus 4pt minus 2pt}{0pt plus 2pt minus 2pt}

\begin{document}

\title{\textbf{Informe de evaluación de ruido debido a las operaciones de vuelo del aeropuerto\\Adolfo Suárez Madrid-Barajas}}
\author{Javier Rodrigo López}
\date{\today}
\maketitle

\pagenumbering{gobble}

\fancypagestyle{firststyle}
{
    \fancyhead[L]{Acústica Ambiental}
    \fancyhead[R]{\includegraphics[width=0.3\linewidth]{Imágenes/ETSIST.png}}
    %\fancyfoot[R]{\includegraphics[width=0.08\linewidth]{Imágenes/Plantilla_IAC.png}\\Departamento de Ingeniería Audiovisual y Comunicaciones}
}

\thispagestyle{firststyle}

% \newpage

\fancyhead{}
\pagestyle{fancy}

\pagestyle{fancy}
\fancyhead[L]{Acústica Ambiental}
\fancyhead[R]{Curso 2024/25}

%\newpage
%\tableofcontents
%\listoffigures
\setcounter{figure}{0}
\setlength{\parskip}{0.5em}

\hypersetup{
    citecolor=black,
    filecolor=black,
    linkcolor=black,
    urlcolor=blueX
}

\tableofcontents
\listoffigures

% IMPRIME ACRÓNIMOS, SIGLAS Y GLOSARIO
%\glsaddall
%\newpage
%\printnoidxglossaries

\newpage

\pagenumbering{arabic}
\setcounter{page}{2}

\section{Introducción}

Este informe forma parte de la Práctica 1 de la asignatura de Acústica Ambiental, impartida por el departamento de Ingeniería Audiovisual y Comunicaciones de la Escuela Técnica Superior de Ingenieros de Telecomunicación de la Universidad Politécnica de Madrid.

El objetivo de este informe consiste en evaluar el ruido producido por las operaciones de vuelo del aeropuerto Adolfo Suárez Madrid-Barajas, motivado por la apertura de una nueva pista. Se han utilizado las normas \cite{ISO1996-2:2020} y \cite{ISO20906:2009} para la medición y evaluación del ruido.

\section{Ensayo}
\section{Equipos y materiales}
\section{Mediciones y resultados}
\section{Conformidad con la legislación vigente}


\nocite{*}
\newpage
\bibliography{Bibliography}
\bibliographystyle{IEEEtr}

\end{document}
