\documentclass[11pt]{article}
\usepackage[spanish, es-tabla]{babel}

\usepackage{amsmath}
\usepackage{amssymb}
\usepackage{amsthm}

\usepackage{fontspec}
\usepackage{multicol}
\usepackage{graphicx}
\usepackage{float}
\usepackage[center]{caption}
\usepackage{graphicx}
\usepackage{listings}
\usepackage[dvipsnames, table]{xcolor}
\usepackage{fancyhdr}
\usepackage{titling}
\usepackage[fixlanguage]{babelbib}
\selectbiblanguage{spanish}
\usepackage{subcaption}
\captionsetup{labelfont=bf}
% \captionsetup[table]{labelfont={bf},name={Tabla},labelsep=period}
\usepackage{hyperref}
\definecolor{blueX}{HTML}{3f84e4}
\setmonofont{SFMonoRegular.otf}
\usepackage[margin=2.8cm]{geometry}
\hypersetup{
    colorlinks,
    citecolor=black,
    filecolor=black,
    linkcolor=black,
    urlcolor=black
}
\usepackage{siunitx}
\sisetup{
    per-mode = fraction,
    detect-all,
    exponent-product = \cdot
}
\usepackage{enumitem}
\usepackage{pdflscape}
\usepackage[stylemods,style=super, nogroupskip, toc=false, hyperfirst=false, nonumberlist]{glossaries-extra}

\setlength{\parindent}{0pt} % To avoid indentation
\setabbreviationstyle[acronym]{long-short}
\makenoidxglossaries
\newglossary{difusion}{difusionin}{difusionout}{Glosario}
\loadglsentries{glosario}

% COLUMNAS Y FILAS MULTIPLES DENTRO DE TABLAS
\usepackage{multirow, hhline}
\usepackage{array}
\newcolumntype{L}[1]{>{\raggedright\let\newline\\\arraybackslash\hspace{0pt}}m{#1}}
\newcolumntype{C}[1]{>{\centering\let\newline\\\arraybackslash\hspace{0pt}}m{#1}}
\newcolumntype{R}[1]{>{\raggedleft\let\newline\\\arraybackslash\hspace{0pt}}m{#1}}
\renewcommand{\listfigurename}{Figuras}
\renewcommand{\listtablename}{Tablas}

% \renewcommand{\familydefault}{\sfdefault}
\renewcommand{\thesubsubsection}{\alph{subsubsection}}

\usepackage{titlesec}
\titleformat*{\subsection}{\normalsize\bfseries}
\titleformat*{\subsubsection}{\normalsize\bfseries}

\titlespacing\subsection{0pt}{12pt plus 4pt minus 2pt}{0pt plus 2pt minus 2pt}
\titlespacing\subsubsection{0pt}{12pt plus 4pt minus 2pt}{0pt plus 2pt minus 2pt}

\begin{document}

\title{\textbf{Informe de evaluación de ruido debido a las operaciones de vuelo del aeropuerto\\Adolfo Suárez Madrid-Barajas}}
\author{Javier Rodrigo López}
\date{\today}
\maketitle

\pagenumbering{gobble}

\fancypagestyle{firststyle}
{
    \fancyhead[L]{Acústica Ambiental}
    \fancyhead[R]{\includegraphics[width=0.3\linewidth]{Imágenes/ETSIST.png}}
    %\fancyfoot[R]{\includegraphics[width=0.08\linewidth]{Imágenes/Plantilla_IAC.png}\\Departamento de Ingeniería Audiovisual y Comunicaciones}
}

\thispagestyle{firststyle}

% \newpage

\fancyhead{}
\pagestyle{fancy}

\pagestyle{fancy}
\fancyhead[L]{Acústica Ambiental}
\fancyhead[R]{Curso 2024/25}

%\newpage
%\tableofcontents
%\listoffigures
\setcounter{figure}{0}
\setlength{\parskip}{0.5em}

\hypersetup{
    citecolor=black,
    filecolor=black,
    linkcolor=black,
    urlcolor=blueX
}

\tableofcontents
\listoffigures

% IMPRIME ACRÓNIMOS, SIGLAS Y GLOSARIO
%\glsaddall
%\newpage
%\printnoidxglossaries

\newpage

\pagenumbering{arabic}
\setcounter{page}{2}

\section{Introducción}

Este informe forma parte de la Práctica 1 de la asignatura de Acústica Ambiental, impartida por el departamento de Ingeniería Audiovisual y Comunicaciones de la Escuela Técnica Superior de Ingenieros de Telecomunicación de la Universidad Politécnica de Madrid.

El objetivo de este informe consiste en evaluar el ruido producido por las operaciones de vuelo del aeropuerto Adolfo Suárez Madrid-Barajas, motivado por la apertura de una nueva pista. Se han utilizado las normas \cite{ISO1996-2:2020} y \cite{ISO20906:2009} para la medición y evaluación del ruido.

\section{Ensayo}
\section{Equipos y materiales}
\section{Mediciones y resultados}

\begin{table}[htbp]
    \centering
    \caption{Indicadores de ruido de las aeronaves registradas}
    \begin{tabular}{|l|l|c|c|c|c|} \hline
        Evento                                              & Hora de inicio     & $T \, (\unit{\s})$ & $L_{Aeq,T}$ & $L_{AE}$ & $L_{Amax}$ \\ \hline
        \rowcolor[rgb]{ .867,  .922,  .969} Aeronave 36R-1  & 20/02/2023 8:37:28 & 56                 & 69.8        & 87.2     & 74.3       \\ \hline
        \rowcolor[rgb]{ .867,  .922,  .969} Aeronave 36R-2  & 20/02/2023 8:39:00 & 16                 & 77.1        & 89.1     & 80.3       \\ \hline
        \rowcolor[rgb]{ .867,  .922,  .969} Aeronave 36R-3  & 20/02/2023 8:41:26 & 16                 & 78.0        & 90.1     & 81.6       \\ \hline
        \rowcolor[rgb]{ .867,  .922,  .969} Aeronave 36R-4  & 20/02/2023 8:52:20 & 18                 & 77.7        & 90.2     & 81.6       \\ \hline
        \rowcolor[rgb]{ .867,  .922,  .969} Aeronave 36R-5  & 20/02/2023 8:55:12 & 15                 & 80.8        & 92.6     & 84.5       \\ \hline
        \rowcolor[rgb]{ .867,  .922,  .969} Aeronave 36R-6  & 20/02/2023 8:56:35 & 18                 & 77.6        & 90.2     & 81.7       \\ \hline
        \rowcolor[rgb]{ .867,  .922,  .969} Aeronave 36R-7  & 20/02/2023 8:58:14 & 30                 & 74.4        & 89.2     & 79.2       \\ \hline
        \rowcolor[rgb]{ .867,  .922,  .969} Aeronave 36R-8  & 20/02/2023 9:00:03 & 20                 & 79.5        & 92.5     & 83.8       \\ \hline
        \rowcolor[rgb]{ .867,  .922,  .969} Aeronave 36R-9  & 20/02/2023 9:04:36 & 32                 & 73.2        & 88.3     & 77.9       \\ \hline
        \rowcolor[rgb]{ .867,  .922,  .969} Aeronave 36R-10 & 20/02/2023 9:06:04 & 19                 & 79.0        & 91.8     & 83.1       \\ \hline
        \rowcolor[rgb]{ .867,  .922,  .969} Aeronave 36R-11 & 20/02/2023 9:13:37 & 21                 & 73.9        & 87.1     & 78.4       \\ \hline
        \rowcolor[rgb]{ .867,  .922,  .969} Aeronave 36R-12 & 20/02/2023 9:18:33 & 19                 & 74.9        & 87.7     & 78.9       \\ \hline
        \rowcolor[rgb]{ .886,  .937,  .855} Aeronave 36L-1  & 20/02/2023 9:02:57 & 33                 & 72.3        & 87.5     & 76.1       \\ \hline
        \rowcolor[rgb]{ .886,  .937,  .855} Aeronave 36L-2  & 20/02/2023 9:15:27 & 47                 & 69.6        & 86.3     & 74.6       \\ \hline
    \end{tabular}
    \label{tab:indicadores}
\end{table}

La duración de los eventos de ruido es un dato inmediato, puesto que solamente es necesario calcular la diferencia en segundos entre la hora de inicio y la hora de fin de cada evento, que han sido seleccionadas manualmente mediante la utilidad de marcadores del programa \textit{Enviro Noise Partner}. El criterio utilizado para delimitar esta zona ha sido la selección de los niveles de ruido situados en un margen de \qty{10}{\dB} con respecto al nivel máximo del evento sonoro.

Cabe mencionar que en varios de los eventos no era posible seguir este criterio, ya que el nivel residual se encontraba a menos de \qty{10}{\dB} del nivel máximo del evento sonoro. Esta situación se describe en la norma \cite{ISO20906:2009}.



\section{Conformidad con la legislación vigente}


\nocite{*}
\newpage
\bibliography{Bibliography}
\bibliographystyle{IEEEtr}

\end{document}
