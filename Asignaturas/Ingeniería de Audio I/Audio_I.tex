\documentclass[10pt]{book}



\usepackage{amsmath}
\usepackage{amssymb}
\usepackage{amsthm}
\usepackage{array}
\usepackage{babelbib}
\usepackage{braket}
\usepackage{caption}
\usepackage{colortbl}
\usepackage{rotating}
\usepackage[table]{xcolor}
\usepackage{color}
\usepackage{enumerate}
\usepackage{esint}
\usepackage{eso-pic}
\usepackage{listings}
\usepackage{lscape}
\usepackage{mathtools}
\usepackage{multicol}
\usepackage{multirow}
\usepackage{siunitx}
\usepackage{subcaption}
\usepackage{subdepth}
\usepackage{tcolorbox}
\usepackage{tikz}
\usepackage{titlesec}
\usepackage{titling}
\usepackage{upgreek}
\usepackage{url}
\usepackage{verbatim}
\usepackage{vwcol}
\usepackage{wallpaper}
\usepackage{xfrac}
\usepackage{physics}
\usepackage[c]{esvect}
\usepackage[utf8]{inputenc}
\usepackage[fleqn]{nccmath}
\usepackage[thicklines]{cancel}
\usepackage[margin=2cm]{geometry}
\usepackage[colorlinks=true,spanish]{hyperref}
\usepackage[oldvoltagedirection]{circuitikz}
\usepackage[greek,spanish,es-tabla,es-nodecimaldot,es-noindentfirst]{babel}
\usepackage[symbol]{footmisc}
\renewcommand{\thefootnote}{\fnsymbol{footnote}}

\sisetup{
  per-mode = fraction,
  detect-all,
  exponent-product = \cdot
}
\hypersetup{
  citecolor = blue,
  linkcolor = blue,
  urlcolor = blue,
  pdfauthor = {Javier Rodrigo López}
}
\captionsetup[figure]{labelfont={bf},name={Figura},labelsep=period}
\captionsetup[table]{labelfont={bf},name={Tabla},labelsep=period}
\titleformat{\section}{\normalfont\Large\bfseries}{\thesection}{1em}{}[{\titlerule[0.8pt]}]
\titleformat{\subsubsection}{\normalfont\normalsize\bfseries}{\thesubsubsection}{1em}{}[{\titlerule[0.05pt]}]
\titlespacing{\section}{0pt}{2\parskip}{\parskip}
\titlespacing{\subsection}{0pt}{\parskip}{0pt}
\titlespacing{\subsubsection}{0pt}{\parskip}{0pt}
\usepackage{enumitem}
\setlist{before={\parskip=3pt}, after=\vspace{\baselineskip}}
\setlength{\parindent}{0pt}
\setlength{\parskip}{0.5em}

\usepackage{booktabs}
\usepackage{bigstrut}

\renewcommand{\vec}{\vv}

% Tipografía
\renewcommand{\familydefault}{\sfdefault}
\renewcommand{\rmdefault}{\sfdefault}

% Para escribir decibelios SPL
\DeclareSIUnit\dbspl{dB\ensuremath{_{\textnormal{SPL}}}}
\DeclareSIUnit\dBlin{dB\ensuremath{_{\textnormal{Lin}}}}
\DeclareSIUnit\dBA{dB\ensuremath{_{\textnormal{A}}}}
 % Se incluye el preámbulo

\title{\Huge Ingeniería de Audio I\\\huge Apuntes de clase}
\author{Javier Rodrigo López}
\date{\today}

\begin{document}

% Este comando crea el título
\maketitle

% Este comando crea el índice
\tableofcontents


\newpage

\chapter{Psicoacústica}

El sonido es entendido como cualquier señal que se produzca en el entorno de \SI{20}{\hertz} a \SI{20}{\kilo\hertz}.

La psicoacústica estudia lo que escuchamos subjetivamente.

La localización espacial del sonido es controlada por los parámetros TD (Time Differencia) y LD (Level Difference). Se basa en la diferencia de distancia que recorre las ondas de la fuente sonora a un oído y otro.

Dos señales son \textbf{coherentes} son aquellas que son iguales o tiene alguna diferencia independiente de la frecuencia (algún retraso, desfase, diferencia de nivel...).

Cuando reproducimos una señal por un altavoz, es una señal monofónica. Si añadimos otro altavoz, la señal reproducida también podría ser monofónica. Al hablar de señal \textbf{estéreo} nos referimos a dos altavoces que reproducen dos señales no coherentes.

\subsection{Ley del primer frente de onda}

La señal que llega con mayor nivel y la que llega primera (menor retardo) es la que entendemos como posición real (concretamente, la posición del altavoz que la emite). La ley del primer frente de onda no sirve a partir del umbral de eco, que depende del tipo de señal que estemos escuchando.

Cuando escuchamos dos señales coherentes en dos altavoces lo percibimos como un único sonido. Si no son coherentes, los entendemos como sonidos distintos.

\chapter{Técnicas de sonido multicanal}

\section{Codificación matricial}

Se realiza cuando se graban multiples canales de audio para un número más limitado de altavoces.

\section{Codifcación discreta}

Como la mayoría de los sistemas exige compatibilidad estéreo, se deben tomar algunas medidas de \textit{Down-Mixing}.
\begin{itemize}
  \item Generando una mezcla independiente para cada sistema de reproducciónn deseado.
  \item Generar algún algoritmo de mezcla, que puede empeorar la calidad de la mezcla significativamente.
\end{itemize}


\end{document}
